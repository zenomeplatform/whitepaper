\section{Use cases}
\subsection{Individual user }
For the individual users there is an opportunity to obtain their genomic information and to turn it into a source of income. The combination of the genome and its interaction with the environment is a valuable information resource. Our platform will allow the user to safely manage this resource.

The platform provides ability to securely store and share genetic information, allowing users to receive a variety of genetic services. Here are a few examples:
\begin{itemize}
\item reports and recommendations on nutrition, risks of diseases, cosmetology, diet, fitness
\item search for relatives and ancestry clarification
\item dating services
\item individual selection of clothing, shoes, home climate setup, travel destinations and areas of residence
\item different variations of the genetic reports for a group of individuals, for example sports teams or working groups
\item almost every aspect of human live is influenced by genetics, so let us see what new companies can come up with using our platform
\end{itemize}

Besides the plurality of services, a user is given an opportunity to make profit from his or her genetic uniqueness by providing questionnaire data to the companies for research purposes. Thus, individual data together with genetic information become an analogue to commodities or mineral resources

\subsection{Health}
Modern healthcare and personalized medicine cannot be imagined without the use of genomic technologies. The platform will allow patients to securely share genetic information used in the clinic with medical personnel:
\begin{itemize}
\item the individual dosage and intolerance to the drugs (for example, individual dosing of the anticoagulant warfarin based on the genetic characteristics)
\item personal acceptable ranges of biochemical body parameters (for example, PSA marker)
\item genetic predisposition to various diseases (for example, a high risk of macular degeneration and the need for additional research and prevention)
\item Transplantation and organ donation. Users may securely share the information regarding the type of their HLA antigens that determines the compatibility between individuals during transplantation. Thus, it will be possible to create a secure database of donors and volunteers to save lives through transplantation.
\end{itemize}

\subsection{Company}
There are two types of Companies, first of which provides users with services based on genomic data, while the second is interested in obtaining genetic data from users to conduct their own research.

First type is described in user's use cases zone. Second type can be described as buying users genomic data to conduct their own research and to improve the consumer properties of products, genetic targeting of products and advertising, some examples of which are:
\begin{itemize}
  \item For example, a pharmaceutical company is planning to release a new drug that acts against the mutant cancer protein. Company may find system users, who survived the disease, to pay them for genetic data and to get the frequency of mutations in a gene encoding the protein that is targeted by active substance
  \item consumer company plans to enter a new market and needs to test how users perceive the taste of the product. It is known that some flavoring causes resentment among carriers of a particular genetic variant of the gustatory receptor gene. The company sends over a network an offer to study the carriers of this genetic variant and picks up another flavoring, or finds out the frequency of this genetic variant in different markets and thus targets the product for different markets.
 \end{itemize}

\subsection{The scientific community}
For the scientific community the system opens the possibility of storing, sharing and performing research with various genomic data. Because the platform is not limited to working with human genomes, it can be used for the secure storage and processing of genomic data, for example, relevant to agriculture (plants, animals, microorganisms).

In general, the presence of ecosystems leads to the enrichment of the scientific community through access to general population data, even without reference to the individual questionnaires. In addition, with the consent of the users, they can also become a part of scientific research.

The platform also provides distributed computational power, access to which will allow to process large volumes of genetic data (similar to AWS adapted to work with genetic data).
