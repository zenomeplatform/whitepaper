\section{Low-Level Interoperation Layer}

\subsection{Distributed P2P network}
A basis of the platform is P2P network based on Kademlia protocol. The network involves the computers of users who have installed Zenome software.

\textbf{Node} --- is a node of distributed P2P network that represents a user's computer with Zenome software installed on it.

\begin{note}
 Consequently, an overlay network is constructed between the devices participating in the network. It represents a virtual network in which to each participant a `NodeId` is assigned, that has no relationship with real IP-address of the device. Each node stores the list of “proximal” nodes, where the distance between nodes is calculated based on nodes’ `NodeId` and is not related to common distance.
\end{note}

The nodes store the data by using distributed hash tables.


\subsection{Distributed network implementation characteristics}

A modified protocol specification is used. Main differences are the following:
\begin{itemize}
\item Nodes can exchange arbitrary messages between each other. A node is able to transfer a message to other node knowing its `NodeId` only.
\item Several hash tables may exist within the network. Hash table is identified by string key.
\item Different policies of value storing and deleting can be set for different tables.
\item The data transferred from one node to another are encrypted (see below).
\end{itemize}

\subsection{Message exchange in a distributed environment}
Thus, participants of peer-to-peer network are able to exchange the following messages:

\begin{tabular}{r|p{13cm}}

  \texttt{PING} & \small Verify that a node is still alive\\

  \texttt{STORE(T,K,V)} & \small Store the value `V` by the key `K` in the table `T` in a node receiving the message\\

  \texttt{FIND\_NODE(N)} & \small A node receiving the message will sent the data regarding the nodes which are closest to the node `N` among the nodes it knows.\\

  \texttt{FIND\_VALUE(T,K)} & \small If the pair `(K,V)` is stored in a receiving node, send the value `V`, else send the data regarding known nodes which are “closer” to file.\\

  \texttt{SEND(M,N,D?)} & \small Sends the message `M` that can also contain the data `D` to the node `N`. `FIND\_NODE` is used to locate the node.

\end{tabular}

\begin{note}
Note: this interaction level is a transport level of the platform.
\end{note}



%\subsection{Low-Level Interoperation Subsystem}
%+ Internal state specification
%
%|>| **Internal State**    | Low-Level Interoperation |
%| ---------- | ----------- | ------------ |
%| **Data**   | **Type**    | **Description**
%|   Data     | `Map<Key,DHT>`| A set of distributed tables |
%
%+ Lists of all interfaces, actions and events
%
%|>| **Interfaces**       | Low-Level Interoperation |
%| ---------------- | ---------- | ------------------------------------------- |
%| **Name**         | **Caller** |  **Description**
%| `Send(N,M,D)`    | `any` at `Node`  | Send message
%| `OnEvent(E,M,H)` | `Node` | When the event `E` fires, which message is exactly `M`, execute the handler `H` with transferring `D` as argument.

%|>| **Events** | Low-Level Interoperation |
%| ------------ | ---------- | ------------ |
%| **Name**     | **Data**   |  **Description**
%| `Message`    | `(N,D)`    | New message has been received
