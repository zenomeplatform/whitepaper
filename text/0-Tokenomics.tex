\chapter*{Introduction: Genome Tokenomics}
\addcontentsline{toc}{chapter}{Genome Tokenomics}

\begin{overview}
  In the following a general description will be presented of the Zenome platform. The need for tokens and benefits for the prospective investor are discussed.
\end{overview}

For the most part, genomic information is stored in databases, financed by governments or large corporations. Individually, each database contains  insufficient data to make the quantum leap towards an era of genomics and precision medicine. At the same time, each database contains so much information that it's impossible for a single company to process all of that.

It appears that the exchange of genetic information is of crucial importance. The prospective genetic market must ensure protection from possible misuse and genetic discrimination in particular. It is particularly important to maintain transparency and equal access to this market.

Global exchange of genetic information should address the following issues:
\begin{itemize}
  \item The fragmentation of genetic data.
	\item The limited access of scientists, medics and companies to genetic data.
	\item Low affordability of genetic testing.
	\item The lack of privacy of those who agreed to open access sharing of their genomic data.
	\item Insufficient computing resources.
\end{itemize}

Zenome aims at creating the personal genomics infrastructure, that would enable participants to:
\begin{itemize}
  \item Upload genetic information and take control of it.
	\item Securely store own genetic information.
	\item Make profit by selling access to genetic data or part of it.
	\item Undergo genetic testing in exchange to the right to use genetic information.
	\item Get individual dietary recommendations or personal training program based on genetic makeup.
	\item Make use of other genetic services.
\end{itemize}
The principal customers of genetic information are companies interested in genetic targeting such as Google, Facebook, Unilever and pharmaceutical companies.

% Проблема нехватки вычислительных мощностей может быть решена за счет привлечения майнеров для выполнения полезной работы в геномной индустрии.

Inside Zenome Platform different types of information, namely genomic, personal and financial data, are inextricably intertwined. The specific nature of each type determines the way to store information of that kind. Financial data, which includes records of transactions, is stored on blockchain. Anonymized genomic data is stored on the distributed network. Participant's personal data is kept on their own computers only. Treating data of different kind diffently provides privacy as well as scalability of the system.

Since all data transactions, including buying and selling data, are governed by smart-contracts, reflecting the decentralized nature of the platform, interactions can only include balances stored on blockchains. Using any of pre-existing tokens for this purpose would result in unreasonable dependence on the valuation of external token or coin. Thus, a separate utility token should be issued to power economic interactions on the platform. This, in particular, means that you cannot buy a genetic data with <<normal>> money, you have to obtain tokens first.

Zenome DNA (ZNA) is an utility token on Zenome platform. The valuation of ZNA is tied to the success of the platform.

In this whitepaper we will discuss in further detail the most pressing problems in the area of genomics, as well as the solution Zenome platform involves.
