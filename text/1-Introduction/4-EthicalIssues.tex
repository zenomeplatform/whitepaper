\section{Ethical Issues of personal genomics}
\begin{overview}
  In this section, ethical problems associated with the development and ubiquitous distribution of genomic technologies are considered, such as privacy, public databases, open access for researchers, possible misuses and threats to personal liberties due to the expansion of genomics.
\end{overview}



\begin{note}
  In the present white paper, we provide only a brief review of the main problems and perspectives of the genomic industry.
  For a more detailed discussion of privacy and data security problems, see \url{https://www.smeal.psu.edu/fcfe/documents/innovations-in-medical-genomics-pdf}.
\end{note}

\subsection{Privacy}

Personal genomic information is very sensitive for many people. However, many people do not fully understand that, based on their genomic information, it is possible to determine their lifespan, propensity to make emotional decisions (manipulability of decision-making), likelihood of developing various mental diseases and risk of sudden death due to, for example, heart arrhythmia.

Such information could be disadvantageous for job recruitment, election participation, and medical insurance pricing. There is also the possibility that a bad actor who knows the sequence of a genome could leave fragments of DNA identical to that genome at, for example, the location of a terrorist act to frame or illegally accuse someone. One could be denied medical treatment (or required to pay a higher fee) or barred from obtaining a desired job.

Corporations and governments could deliberately influence one’s decisions and purchases using their knowledge of “weaknesses” in one’s genomic information. Thus, the protection of genomic data privacy is necessary to protect the equal rights of various categories of people.

At the same time, some studies have been performed that allow the identification of individuals’ identities based on their anonymous genomes
\cite{Gymrek321,nature:the-genome-hacker}.

Moreover, some companies (\url{http://www.humanlongevity.com/media/}) possess machine learning-based algorithms that can accurately reconstruct the appearance of an individual using only his or her genomic data\cite{TED:FaceFromDNA}.

\subsection{Publicity}
Such an approach will enable the development of preventive medicine through which the analysis of large amounts of data allows the prediction of disease development (before its occurrence), enabling actions that increase lifespan and improve quality of life\cite{jpm2030093} as well as the identification of donors worldwide to meet various medical needs. Unfortunately, no valid solutions currently exist for the public use of genomic information while maintain individual privacy. However, some startups working in this field using blockchain technology should be mentioned.

\href{https://www.encrypgen.com/}{Encrypgen}, a startup that has recently conducted an ICO, described the existing problems and the relationship between privacy and publicity using blockchains\cite{forbes:missinglink}. However, the white paper associated with this project lacks a description of a technical implementation that would solve the problem of privacy and availability.

Another project in this field is the DNAbits startup\cite{coindesk.com:DNAbits}, whose founder, Dror Samuel Brama, has patented a general approach to data storage and transfer using blockchain technologies\cite{brama2015method}. However, this company has not technically implemented its concept during the past three years.

\subsection{The right to own genomic data?}
Currently, there is no legislative definition of the right to possess one’s own genetic information. In some developed countries, including the USA, Germany, and Austria, citizens do not have the right to access and possess their genetic data in the context of its interpretation\cite{goodreads.com:24714901}.  An agent, represented by a physician or medical center with the right to provide such information, is needed. This path is used by the companies \href{https://www.pathway.com/}{Pathway Genomics} in the USA and CeGaT in Germany (\url{http://www.cegat.de/en/}).

To undertake genetic analysis, the advice of a physician who could be a provider of genetic testing is needed, and only this physician has a right to interpret the information provided by genetic analysis.

In the USA, there are service providers in the field of “genetics for fun,” such as the companies \href{https://www.23andme.com}{23andMe} and \href{https://www.ancestry.com/}{Ancesty.com}, which sell genetic tests directly to the end customer, but these companies can only provide information on ethnic origin and certain health-related characteristics (for example, sports characteristics) and lack the permission to provide most medically valuable information. These restrictions, imposed by regulators such as the FDA, do not hinder the ability of 23andMe to sell access to genetic data to large pharmaceutical companies. Some such deals are known: a deal was made with Genentech (a subdivision of the pharmaceutical giant Roche) for \$60 million\cite{forbes:genentech} for a study of Parkinson’s disease, and a deal was made with another large pharmaceutical company, Pfizer, for a study of inflammatory bowel diseases (e.g., Crohn’s disease)\cite{genomeweb.com:23andme-pfizer}. Some reports also claim that 23andMe has had negotiations with Novartis over an Alzheimer’s disease study\cite{alzforum.org:23andme-alzheimers}.

Thus, we currently give large companies the right to manage our genomic information, to store it, and to profit from it. Corporations, behind the veil of good intentions, monopolize genomic big data, and we cannot predict how this monopoly will influence future drug prices and medicinal discoveries.

\subsection{The right to access genomic information}

Another ethical issue should be discussed; above, we noted that privacy violations and access to genomic information could be used illegally, for example, by an employer. One could be fired or denied a job promotion based on genetic information. For those with jobs related to the safety of people and systems, such as truck or bus drivers, pilots, atomic power station operators, or people with other similar occupations, health status is critically important, and genomic information could prevent an accident or even a disaster. In some professions, a potential danger could threaten the worker rather than bystanders, such as a coal or diamond miner with lung problems. For these cases, a discussion involving professionals and experts as well as the general public is necessary to develop legal standards regulating the use of genomic information by employers.
