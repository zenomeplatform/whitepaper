\section{Background}
%-------------------------------------------------------------------------------
\begin{overview}
In this section, definitions of the terms “genome” and “genomics” are given. The history of the first genomic sequencing efforts and the emergence of high-throughput sequencing technology (NGS) are considered. The main manufacturers of reagents and equipment used to obtain genomic data are described. Issues regarding accumulating genomic data and reducing the cost of analyses are considered. A review of current genomic databases is provided.
\end{overview}
%-------------------------------------------------------------------------------


%-------------------------------------------------------------------------------
\subsection{Genome}
\textbf{The genome} is the complete set of genetic instructions found in a cell \cite{genome.gov:glossary:genome}.

The genome contains biological information necessary for the development and functioning of an organism. The human genome consists of linear double-helical DNA molecules organized into 22 pairs of chromosomes plus two sex chromosomes – X and Y. All information contained in a genome is encoded using quaternary code through a sequence of 4 nucleotides designated A, T, C, and G. The term "to read a genome" means “to determine a nucleotide sequence by a sequencing process” \cite{clinchem:61:9:1207}.

The individual sequence of a genome defines a variety of organismal features, including appearance, susceptibility to certain diseases, athletic ability, metabolism, nutritional preferences, compatibility with sexual partners (the ability to conceive children), and many more.

\subsection{The International Human Genome Project}
The International Human Genome Project\footnote{\url{https://en.wikipedia.org/wiki/Human_Genome_Project}}  was launched under the supervision of the NIH (National Institutes of Health) in 1990 to determine the complete sequence of the haploid human genome. The initial project leader was one of the discoverers of the structure of DNA, Nobel prize winner James Watson

A draft sequence of the human genome was completed in the middle of 2000 and published in the beginning of 2001 in the journal Nature. The cost of this international project completed with public funding was approximately \$3 billion. In 1998, a private company, Celera Genomics, joined the race to sequence the human genome. The leader of the private project, which was developed in parallel with governmental institutions, was famous scientist and entrepreneur Craig Venter, who managed to raise \$300 million in private investments for Celera’s project. By using the new shotgun sequencing approach and more productive computational methods, the sequence of Craig Venter’s genome was published almost simultaneously with the data produced by the international consortium in 2001\cite{Venter1304} in the journal Science. The "full" human genome was published in 2007, and some human genomic regions that are difficult to sequence remain unknown.

\subsection{Development of genome analysis}
Extensive investments, a large number of outstanding participants from the scientific community, and competition among private and public organizations have provided considerable impetus for developing genome analysis technologies. As a result, modern sequencing technologies such as NGS (next-generation sequencing)\footnote{\url{https://www.ebi.ac.uk/training/online/course/ebi-next-generation-sequencing-practical-course/what-you-will-learn/what-next-generation-dna-}} have emerged together with a new branch of science called bioinformatics, a young field of research at the intersection of mathematics, IT, and biology, that develops techniques and algorithms for the analysis of large biological datasets in productive and computationally effective ways.



\begin{table}[h!]
  \caption{Equipment developers and reagent suppliers for genomic sequencing. Market capitalization taken from \href{https://finance.yahoo.com/}{Yahoo Finance}.}
  \label{table:developers}
  \begin{tabular}{|p{4cm}|p{7.5cm}|c|c|}

\hline\bf Company
  & \rotatebox{0}{\bf Products}
  & \rotatebox{90}{\bf Capital-}
    \rotatebox{90}{\bf-ization}
  & \rotatebox{90}{\bf Country}  \\

\hline Illumina
  & \small Hardware, reagents, consumables, \newline
    \small software
  & \small 28.06\,B & USA \\

\hline Thermo Fisher \newline Scientific
  & \small Hardware, reagents, consumables, \newline
    \small software (a part of business)
  & \small 68.98\,B & USA \\

\hline Oxford Nanopore Technologies
  & \small Hardware, reagents, consumables
  & \small 534.41\,M & UK \\

\hline Pacific\newline BioScience
  & \small Hardware, reagents, consumables
  & \small 436,93\,M & USA \\

\hline Roche
  & \small Hardware, reagents, consumables, \newline
    \small software (a part of business)
  & \small 213.44\,B & \footnotesize Switzerland \\

\hline Agilent\newline Technologies
  & \small Hardware, reagents, consumables, \newline
    \small software (a part of business)
  & \small 19.32\,B & USA \\
\hline

\end{tabular}

\end{table}

The emergence of second and third generation sequencing technologies (NGS) has led to a strong reduction of genome analysis cost. While even in 2009 the cost of a full analysis of a genome was about 100,000 USD, currently the average price for the same analysis has dropped down to approximately less than 1,000 USD (see Fig.~\ref{fig:Moore} and Table~\ref{table:genomeanalysistypes}).

\blfootnote{\url{https://en.wikipedia.org/wiki/Moore's_law}}
\begin{figure}[h!] \centering
  \begin{tikzpicture}
\pgfplotstableread{
  mm  yy   megabyte      full
  9  2001   5292.390    95263072
  3  2002   3898.640    70175437
  9  2002   3413.800    61448422
  3  2003   2986.200    53751684
  10 2003   2230.980    40157554
  1  2004   1598.910    28780376
  4  2004   1135.700    20442576
  7  2004   1107.460    19934346
  10 2004   1028.850    18519312
  1  2005    974.160    17534970
  4  2005    897.760    16159699
  7  2005    898.900    16180224
  10 2005    766.730    13801124
  1  2006    699.200    12585659
  4  2006    651.810    11732535
  7  2006    636.410    11455315
  10 2006    581.920    10474556
  1  2007    522.710     9408739
  4  2007    502.610     9047003
  7  2007    495.960     8927342
  10 2007    397.090     7147571
  1  2008    102.130     3063820
  4  2008     15.030     1352982
  7  2008      8.360      752080
  10 2008      3.810      342502
  1  2009      2.590      232735
  4  2009      1.720      154714
  7  2009      1.200      108065
  10 2009      0.780       70333
  1  2010      0.520       46774
  4  2010      0.350       31512
  7  2010      0.350       31125
  10 2010      0.320       29092
  1  2011      0.230       20963
  4  2011      0.190       16712
  7  2011      0.120       10497
  10 2011      0.090        7743
  1  2012      0.090        7666
  4  2012      0.070        5901
  7  2012      0.070        5985
  10 2012      0.070        6618
  1  2013      0.060        5671
  4  2013      0.060        5826
  7  2013      0.060        5550
  10 2013      0.060        5096
  1  2014      0.040        4008
  4  2014      0.050        4920
  7  2014      0.050        4905
  10 2014      0.060        5731
  1  2015      0.040        3970
  4  2015      0.050        4211
  7  2015      0.015        1363
  10 2015      0.014        1245
} \mytable
  \begin{semilogyaxis}[
    width=16cm, height=7cm, compat=newest,
    % TIKS -----------------------------------
    xtick={2001,2002,...,2015},
    ytick={1000, 10000, 100000, 1000000, 10000000, 100000000},
    yticklabels={1K, 10K, 100K, 1M, 10M, 100M},
    x tick label style={%
      /pgf/number format/set thousands separator={},%
      font=\small, xshift=-0.4ex, yshift=-2mm,anchor=west },
    y tick label style={font=\small, xshift=0.7ex },
    % LINES -----------------------------------
    axis x line = bottom,
    axis y line = middle,
    ylabel={Sequencing Costs, USD}, y label style={anchor=south west},
    % GRID -----------------------------------
    grid=both,
          grid style={line width=.1pt, draw=gray!10},
    major grid style={line width=.2pt, draw=gray!50},
    % Limits -----------------------------------
    ymin=500,  ymax=500000000,    xmin=2000.5,  xmax=2016.2,
  ]
    \addplot[color=gray, mark=*, mark size=0.7]
      table [ x expr={\thisrow{yy}+(1/12)*\thisrow{mm}}, y=full ] {\mytable};
    \addplot [no markers, color=black, domain=2001.8:2015] {exp(-0.36*(x-2001.8)+18.42)};
    \end{semilogyaxis}
\end{tikzpicture}

  \caption{Moore's Law and cost reduction of genomic analysis. A significant drop in prices in 2008 was due to the advent of next generation sequencing technologies (NGS)\cite{genome.gov:sequencingcostsdata}.}
  \label{fig:Moore}
\end{figure}


\subsection{Impact on other sciences}
The development of genomics (the field of science studying various genomes) has led to the transformation of many scientific fields, from biology and anthropology to medicine and even social sciences. A number of top commercial companies such as Google, Apple, IBM, Amazon, and Alibaba have set the objective of using genomics to adjust their products and services according to the genomic profiles of their customers. Such adjustments will allow these companies to fine tune user relations and to predict their customers’ needs and potential activities
\footnote{\url{https://www.smeal.psu.edu/fcfe/documents/innovations-in-medical-genomics-pdf}}.

\subsection{Genetic databases}
The reduction in the cost of sequencing has led to an exponential increase in available genomic data. For example, a complete human genome in so-called "raw data" format can represent 50 GB to 2 TB of data (depending on the sequencing depth required). To store such a large amount of genomic data, special genomic databases have been created containing various types of data, such as raw data obtained from genomic sequencers ("reads" or "readings"), sequences of genes and proteins, sets of coding regions of a genome called exomes, and even the sequences of whole genomes (scaffolds); some of these databases contain clinically relevant information as well as the relationships between genetic traits and diseases. Most of these databases are centrally managed and financed by governments or large corporations. Scientists worldwide are involved in the addition of new data to these databases, enabling rapid updating and synchronization. In Table~\ref{table:databases}, some such well-known databases are described.


\begin{table}[h!] \centering
  \caption{Genomic databases}
  \label{table:databases}
  \def\Field #1 => #2;{ \hfill \small #1: & \small #2 \\[-5pt] }
\def\Line#1=> #2;{ & \small #2 \\[-5pt] }


\def\Record #1 at #2 [#3]{
\hline  \multicolumn{2}{|c|}{ #1 \hfill \url{#2}  }
\\ \hline  #3 \\[-3mm]
}

\begin{tabular}{p{2cm}p{15cm}}

\Record GenBank at http://exac.broadinstitute.org [
\Field Owner    => NCBI-NIH, USA;
\Field Product  => Genome sequences database;
\Field Stored   => More than 199,341,377 different genome sequences;
]

\Record ExaC at www.ncbi.nlm.nih.gov/genbank [
\Field Owner   => Broad Institute of MIT and Harvard, USA,
                 ODC Open Database License (ODbL);
\Field Product => Exome Aggregation Consortium;
\Field Stored  => 60,706 human exome samples/sequences;
]

\Record UniprotKB at www.ebi.ac.uk/uniprot/ [
\Field Owner   => EMBL-EBI, SIB, PIR, UK, Switzerland, USA ;
\Field Product => Open Knowledge Base. Manual expert curation. Proteins and genes sequences.;
\Field Stored  => More than 555,100 manually reviewed and annotated record ;
]

\Record ClinVar at https://www.ncbi.nlm.nih.gov/clinvar/ [
\Field Owner   => NCBI-NIH, USA ;
\Field Product => freely available archive for interpretations of
 clinical significance of genomic variants for reported condition ;
\Field Stored  => >158 000 submitted interpretations, representing >125 000 variants. ;
]


\Record HGMD at http://www.hgmd.cf.ac.uk/ac/index.php [
\Field Owner   => QiaGen ;
\Field Product => Commercial database ;
\Field Stored  => 208,368 human mutation records with annotations ;
]

\Record SNPedia at https://www.snpedia.com/index.php/SNPedia [
\Field Owner   => Open database ;
\Field Product => SNPedia is a wiki investigating human genetics ;
\Field Stored  => 107,073 SNPs and linked records ;
]

\Record 1000 Genomes Project at www.1000genomes.org [
\Field Owner    => EMBL-EBI, Wellcome Trust ;
\Field Goal     => find most genetic variants with frequencies of at least 1\%;
\Field Stored   => More than 2,504 Genome samples/sequences ;
]

\Record 100000 Genomes Project at http://www.genomicsengland.co.uk/ [
\Field Owner   => NHS, Government of UK ;
\Field Product => UK government database containing a sequence of 100,000 genomes ;
\Field Stored  => 32,642 Whole genome sequences ;
]

\end{tabular}

\end{table}

The majority of such databases are hosted in developed countries and are centrally managed and controlled by governments. Access to some of these databases is restricted, even for the scientific community, or is limited by commercial subscriptions. Although the founders of genomic databases claim they securely and anonymously store genomic data, in reality, the data stored are only pseudonymous, as in some cases, individuals have been identified based on their genomic information\cite{Gymrek321}.
